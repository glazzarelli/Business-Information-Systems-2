\section{Workforce Management Lecture}\label{workforce-management-lecture}

\myfigure{2}{\trimLR}{4cm}{images/00 - WFM.pdf}

\subsection{Extended ERP and Workforce Management}\label{extended-erp-and-workforce-management}

Now, let's move on to the topic of workforce management. Extended ERP is
the focus here. Extended ERP aims to integrate all information flows and
processes involving not only internal actors like employees but also
external players such as clients and partners of the company. This
comprehensive approach ensures efficient interactions between the
company and its stakeholders.

\myfigure{3}{\trimLR}{3.5cm}{images/00 - WFM.pdf}

\subsection{Client Relationships and Transaction Models}\label{client-relationships-and-transaction-models}

Workforce management is closely tied to the relationship between a
company and its clients, particularly during the post-sale phase. This
relationship is governed by a transaction model, which we discussed in
the context of transaction theory. It's important to note that there are
various types of transactions, depending on the nature of the goods or
services involved. However, all transactions ideally consist of four
phases, with the final phase being post-settlement or post-sale
services. Workforce management focuses on these post-sale services and
the ongoing relationship between the company and the client.


\subsection{Post-Sale Services and WFM in Manufacturing}\label{post-sale-services-and-wfm-in-manufacturing}

When discussing workforce management, it is often in the context of the
manufacturing industry. In this industry, companies produce physical
products that are installed at the customer's location. Examples of
these installations include elevators, household appliances like washing
machines and dishwashers, as well as appliances for electricity or gas.
The focus of workforce management in this context is on maintaining
these installations.

Once a customer has purchased an appliance, they may encounter issues
that require maintenance. The manufacturer is responsible for addressing
these maintenance requests. Typically, the customer has already paid for
the appliance, so the transaction has been completed. However,
additional services are needed to resolve any issues with the product.
To provide these services, the company must send specialized employees
to the customer's location.

\subsection{Maintenance Challenges and
    Costs}\label{maintenance-challenges-and-costs}

For example, let's say your elevator is broken. In this scenario,
someone from the company needs to come to your building to fix it.
However, having a workforce that needs to travel to different locations
comes with a significant cost for the company. It would be much more
cost-effective if the client could bring the appliance to the company
for repairs.

To illustrate this, let's consider the example of a broken cell phone.
When your cell phone breaks, you don't have someone come to your place
to fix it. Instead, you go to a shop that specializes in cell phone
repairs. You might go to a shop in Chinatown, for example, where they
provide excellent service. You drop off your phone, go about your day,
and then return to pick it up once it's fixed. This process requires you
to invest your time in physically going to the shop, which can be a
significant amount of time. However, you are not charged for the
traveling costs.

In contrast, when a company needs to send a workforce to your home to
fix a broken appliance, it becomes more complex. The company has to
consider the costs associated with sending their workforce to your
location, including travel expenses and the time required for travel.
These costs are eventually factored into the price of the service, which
you, as the client, will have to pay. Unlike when you take your cell
phone to a shop, where you are not charged for traveling costs, the
company needs to include these costs in their service fees.
Additionally, the company must ensure that they send the right skilled
workers to address the specific issue you are facing.

\subsection{Emergency vs.~Routine
    Maintenance}\label{emergency-vs.-routine-maintenance}

\myfigure{4}{\trimLR}{7cm}{images/00 - WFM.pdf}

And so, when it comes to fixing appliances like elevators, the team
responsible for the repairs needs to have the necessary skills and
competencies. Unlike other appliances that can be taken to the company
for repairs, elevators need to be fixed on-site, which often requires
multiple visits. The first visit is to assess the problem, and
subsequent visits are made with the appropriate team and materials to
fix the issue. This process is complex and involves managing response
times, which is crucial for customer satisfaction and safety.

In the case of elevators, response time is particularly important
because a malfunctioning elevator can pose a safety risk, especially if
someone is trapped inside. For example, if a patient in a hospital is
being transferred between departments and the elevator gets stuck, it
becomes a matter of urgency to rescue them. The level of service
provided in these situations is not only important for customer
satisfaction but also for ensuring safety. Incidents where people are
trapped in elevators and their lives are at risk are rare but
significant. Therefore, the level of service and efficiency in handling
emergencies is crucial.

Maintenance plays a vital role in assessing the quality of service
provided by a company. When an appliance breaks down, it is an
opportunity for the company to demonstrate its commitment to customer
satisfaction. If the company efficiently handles the maintenance
process, customers are more likely to remain loyal. Studies have shown
that satisfied customers who receive efficient maintenance services are
more likely to continue purchasing products from the same company.
Acquiring new customers is challenging, so maintaining the loyalty of
existing customers is essential for long-term competitiveness.

Elevators are a prime example of the importance of maintenance because
they are long-lasting investments. Instead of replacing a broken
elevator, it is more cost-effective to repair it. Therefore, maintenance
becomes even more critical for elevators. However, maintenance requests
can be divided into two types: routine maintenance and emergency
maintenance. Routine maintenance, such as changing the oil in a car
engine, can be planned and scheduled in advance. It is preventive
maintenance that helps avoid breakdowns and can be optimized for
efficiency. On the other hand, emergency maintenance is unpredictable
and requires immediate attention.

\myfigure{5}{\trimLR}{11cm}{images/00 - WFM.pdf}

Routine maintenance is relatively easy to manage and profitable for
companies. It can be planned ahead of time, and the effort required is
known in advance. This type of maintenance is designed to be
straightforward and can be performed without specialized expertise or
spare parts. It is a predictable demand that allows companies to
allocate resources efficiently. In contrast, emergency maintenance is
unpredictable and requires a reactive response.

In summary, the composition of the workforce and their competencies are
crucial for fixing appliances like elevators. Response time is essential
for customer satisfaction and safety, especially in emergency
situations. Maintenance is an opportunity for companies to demonstrate
their commitment to customer service and build loyalty. Routine
maintenance can be planned and optimized, while emergency maintenance
requires immediate attention. By understanding the different types of
maintenance and their importance, companies can ensure long-term
competitiveness and customer satisfaction.

\subsection{Customer Service and Maintenance
    Requests}\label{customer-service-and-maintenance-requests}

In the case of sudden problems that require management, the typical
scenario involves a client calling the company to report the issue. This
asynchronous communication requires a synchronous response from the
company to handle the problem efficiently. However, this type of demand
is difficult to forecast, leading to challenges in planning and
efficiency. Emergency maintenance, especially if it involves safety,
must be prioritized regardless of location. This means that companies
must be prepared to serve clients in remote areas, even if it is less
convenient or profitable.

\myfigure{8}{\trimLR}{5cm}{images/00 - WFM.pdf}

For manufacturing companies, the transaction with the customer typically
ends with the payment, and there is no ongoing contract or relationship.
This can lead some clients, especially businesses, to seek maintenance
services from other providers instead of the manufacturer. This is often
driven by the perception that independent maintenance services are
cheaper than going to the manufacturer. Smaller maintenance services
tend to focus on profitable areas with high concentrations of potential
issues, while leaving less profitable areas to the manufacturer. As a
result, manufacturers are left with the more challenging and less
profitable emergency maintenance tasks.

\myfigure{6}{\trimLR}{6cm}{images/00 - WFM.pdf}

Emergency maintenance is generally not profitable, especially when it
requires sending a workforce to the customer's location. Maintenance
activities can be costly due to the geographical distribution of
customers, the expense of visits, and the difficulty of finding
employees who are skilled at problem-solving. Employees who excel at
fixing appliances may even leave to start their own maintenance
services, becoming competitors to the manufacturer. Additionally,
handling maintenance requests can be challenging, as customers often
express their frustration and anger without providing essential
information about the appliance and its model.

Overall, routine maintenance is typically more profitable than emergency
maintenance. However, the nature of emergency maintenance, with its
generic and often emotionally charged requests, makes it more difficult
to manage efficiently.

\subsection{Optimizing the Maintenance
    Process}\label{optimizing-the-maintenance-process}

To optimize the maintenance process, it is important for customers to provide essential information about the appliance, such as its location and details. However, many customers are not aware of this and may struggle to accurately describe the problem. This is where the call center operators play a crucial role in asking the right questions to gather the necessary information. Unfortunately, level one call center operators often lack the skills to ask precise questions, which can lead to inefficiencies in the process.

\myfigure{9}{\trimLR}{4cm}{images/00 - WFM.pdf}

To address this issue, it is important for organizations to ensure that the call center operators have access to an enterprise resource planning (ERP) system. This system can help them set up appointments and share the workforce's agenda, ensuring that the right resources are available when needed. Additionally, having spare parts readily available is essential to avoid delays in the maintenance process. However, tight service level agreements (SLAs) can pose challenges, as the workforce needs to be adequately sized to handle peak calls, which can increase costs.

\myfigure{7}{\trimLR}{7.5cm}{images/00 - WFM.pdf}

It is common for maintenance to be seen as a non-profitable and challenging process. However, it is crucial to change this mindset and view maintenance as a service that can help build customer loyalty. Outsourcing maintenance may seem like a good solution, but it can have negative repercussions on brand equity if issues arise. Therefore, organizations should carefully consider the level of service provided by maintenance service providers before outsourcing.

The worst possible maintenance process is one where the customer provides limited information about the issue, leading to multiple visits and coordination challenges within the workforce. This not only inconveniences the customer but also increases costs. Additionally, missed opportunities for cross-selling\footnotemark{} can occur during maintenance interventions. This is the perfect time to offer maintenance subscriptions or additional services to customers.

To optimize the maintenance process, it is crucial for call center operators to ask the right questions and gather accurate information from customers. This will help streamline the process, reduce costs, and improve customer satisfaction.

\footnotetext{Cross-selling is a sales strategy where a company encourages customers to purchase additional products or services related to their initial purchase. The goal of cross-selling is to increase the average transaction value and maximize revenue from each customer. This strategy involves suggesting complementary or supplementary items that go hand-in-hand with the customer's original purchase, thereby enhancing their overall experience and meeting more of their needs.}

\subsection{Knowledge Management and Business
    Intelligence}\label{knowledge-management-and-business-intelligence}

\myfigure{10}{\trimLR}{3cm}{images/00 - WFM.pdf}

To ensure efficient maintenance processes, it is important to have an
intelligent Q\&A system in place. This system allows the maintenance
workforce to diagnose problems accurately and quickly. By diagnosing the
problem, the necessary spare parts can be prepared in advance. The ERP
system can be used to check if the spare parts are available in the
physical warehouse and ensure they are loaded onto the truck. This way,
the technician can make an appointment at the right time and avoid any
delays.

Invoicing is another crucial aspect of the maintenance process. It is
essential to make the customer aware of why they are spending money and
provide a breakdown of the costs, including the price of the spare
parts. By doing this, the customer can understand the value they are
receiving for their investment. It is recommended to print the invoice
as soon as possible, preferably during the discussion with the customer,
to ensure accuracy and transparency.

Training the workforce to handle the administrative part of the job is
necessary. Although technicians may not enjoy administrative tasks, it
is important to equip them with the skills to handle invoicing and other
administrative responsibilities. This can be facilitated by using a
simple application that tracks materials used during maintenance
services and simplifies the invoicing process.

\myfigure{11}{\trimLR}{6cm}{images/00 - WFM.pdf}

Implementing knowledge management is crucial for an intelligent Q\&A
system. By analyzing past maintenance interventions and building
diagnostic intelligence, the system can ask the right questions and
provide accurate solutions. Additionally, appliances equipped with
sensors can self-diagnose and provide error codes, making the Q\&A
process even more efficient.

Furthermore, by analyzing data from maintenance services, companies can
gain valuable insights into common issues with different appliance
models. This information can be used to improve product design and solve
recurring problems. However, it is important to note that using
knowledge management to intentionally shorten the lifespan of appliances
for profit is unethical and should not be practiced.

In summary, the key actions to achieve an ideal maintenance process
include implementing knowledge management, using an intelligent Q\&A
system, training the workforce in administrative tasks, and analyzing
data for continuous improvement.

\subsection{Case Studies}\label{case-studies}

\subsubsection{OTIS Elevators Example}\label{otis-elevators-example}

Knowledge management is essential for analyzing data and understanding
maintenance issues. By collaborating with the maintenance team and Research and Development (R\&D) team, you can identify and fix problems before they become emergencies. This
not only helps avoid costly interventions but also allows for the
transformation of emergency maintenance into routine profitable
maintenance. By analyzing information on maintenance interventions and
working with R\&D, you can discover the most common and recurring issues
and implement preventive maintenance to avoid emergencies. This
knowledge can be kept confidential and used to offer a competitive
service in the market.

\myfigure{13}{\trimLR}{7cm}{images/00 - WFM.pdf}
\myfigure{14}{\trimLR}{8cm}{images/00 - WFM.pdf}

A case study that exemplifies these issues is the OTIS Elevator company. They
faced the challenge of unprofitable maintenance services due to
competition. Competitors focused on the easiest and most profitable
interventions in urban areas, leaving OTIS Elevator with elevators in
remote areas that lacked maintenance subscriptions and routine
maintenance. This resulted in numerous emergency calls. Additionally,
OTIS Elevator mismanaged interventions and faced payment issues from
clients. While a small percentage of unpaid interventions can be
managed, if it grows, it becomes a problem as it affects the
profitability of the maintenance processes.

\myfigure{15}{1.5cm}{5cm}{images/00 - WFM.pdf}

To address these challenges, OTIS Elevator decided to change their
maintenance processes and implement knowledge management. This involved
conducting proper business intelligence analysis and data analytics to
uncover the root causes of issues and their relationships. By working
with data scientists and R\&D, they established routine maintenance
interventions that reduced the need for emergency interventions. They
also leveraged their ability to diagnose issues before they occur and
offered highly competitive prices for their maintenance services.

By understanding the issues and their causes, OTIS Elevator was able
to proactively replace parts after three years, resulting in cost
savings for customers and the ability to offer maintenance interventions
at a discounted price compared to the market. Through this process, they
invested in customer relationships and subscription maintenance
services, reducing the need for emergency maintenance by 30 percent and
becoming more competitive in terms of pricing. This demonstrates the
benefits of industrializing processes by standardizing and making them
more efficient.

By utilizing knowledge management and business intelligence, OTIS
Elevator was able to act proactively and avoid problems. This approach
not only improved their maintenance processes but also enhanced their
overall service quality.

\myfigure{16}{\trimLR}{4cm}{images/00 - WFM.pdf}
And then they implemented an industrialized process. The value
proposition became subscribing to their maintenance service, which would
result in a 30 percent reduction in emergency maintenance interventions
and cost savings for customers. This would also improve the quality of
their lives by minimizing the need to handle emergencies. By offering
this service, they could shift the blame to competitors if any issues
arose due to improper routine maintenance. In business contexts, people
are willing to pay to avoid problems, making this a strong value
proposition.

\myfigure{17}{\trimLR}{5cm}{images/00 - WFM.pdf}

They successfully sold many maintenance services by
investing in IoT and offering free sensors and diagnostic tools for old
elevators. This allowed them to send the right technicians with the
necessary spare parts at the right time. Ultimately, their solution
extended the lifespan of the equipment and protected the owner's
investment. They positioned maintenance as a service rather than a
costly and unprofitable process.

\myfigure{18}{\trimLR}{6cm}{images/00 - WFM.pdf}

Their web-based system included remote
monitoring capabilities, allowing them to proactively address issues
before they occurred. They would contact customers to schedule
maintenance visits, creating a high level of trust and customer
satisfaction. This approach allowed them to charge premium prices,
similar to a luxury hotel experience.

\myfigure{19}{\trimLR}{11cm}{images/00 - WFM.pdf}

\subsubsection{Utility Company and Smart
    Meters}\label{utility-company-and-smart-meters}

\myfigure{21}{\trimLR}{9cm}{images/00 - WFM.pdf}

Let's discuss the use of intelligent meters by utility companies. These
meters provide real-time information on consumption, allowing for
accurate billing without the need for manual readings. Instead of
sending a workforce to check the meters, utility companies can position
intelligent meters that provide consumption data in real time. This
eliminates the need for manual readings and reduces costs.

\myfigure{26}{\trimLR}{7cm}{images/00 - WFM.pdf}

However, utility companies have not fully utilized the potential of the
information provided by these meters. Currently, they mainly use the
data for self-diagnosis and fixing any issues with the meters. While
this is useful, it does not offer a compelling value proposition to
customers. The meters themselves are simple and do not require frequent
maintenance like elevators, so the potential for additional services is
limited.

The value proposition of these meters is primarily focused on improving
invoicing accuracy. However, utility companies are now reconsidering
their approach due to sustainability concerns and the increasing cost of
energy. They are exploring the idea of using renewable energy sources
and optimizing energy consumption to reduce costs. This could
potentially create a more compelling value proposition for customers.

In conclusion, while utility companies have not fully capitalized on the
potential of intelligent meters, there is ongoing exploration of new
value propositions. The example of the OTIS elevator remains a
favorite case study, highlighting the importance of integrating
maintenance into the overall service cycle.

