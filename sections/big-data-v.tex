\section{Big Data Analytics - V Lecture}

\subsection{Solution and Case Studies Part Two}

\subsubsection{Big Data Project Management}

We previously covered the
importance of defining and selecting use cases, determining the project
theme, and planning the project. Now, let's move on to step four:
defining the technical requirements.

\begin{figure}[!h]
    \centering
    \includegraphics[page=88, trim = 0cm 3cm 1.5cm 4.5cm, clip, width=\textwidth]{images/06 - BIG_DATA.pdf}
\end{figure}

When defining the technical requirements, it's crucial to prioritize the
business needs before considering the technology. It's recommended to
have a clear understanding of what you require during the pilot stage.
This means avoiding the purchase of unnecessary infrastructure and
technologies at the initial stage. Typically, a pilot requires a smaller
infrastructure and a limited set of technologies.

Once the pilot stage is successfully completed, and you are ready to
scale up and deploy the project, that's when you should invest in a
full-fledged infrastructure and architecture that meets your needs. By
following this approach, you can ensure that your big data project
progresses efficiently and effectively.

\subsubsection{Case Study of Targeting}

\begin{figure}[!h]
    \centering
    \includegraphics[page=89, trim = 0cm 3cm 1.5cm 4.3cm, clip, width=\textwidth]{images/06 - BIG_DATA.pdf}
\end{figure}

In this case study, we will explore the concept of targeting, which
encompasses more than just the recommendation engine used in retail.
Targeting involves automating various functionalities using big data and
analytics. While the online recommendation engine is a part of
targeting, there are other functionalities as well.

One example is the assortment optimization engine, which allows
customers to have a personalized shelf where they can easily find
products that are preferred by certain customers or even a specific
customer. Another functionality is the personalized online search
engine, which prioritizes products that are similar to the ones the
customer likes or prefers. Additionally, there is the personalized
shopping list feature, which automatically creates a shopping list based
on the customer's preferences.

\begin{figure}[!h]
    \centering
    \includegraphics[page=95, trim = 1.5cm 3.5cm 1.5cm 4cm, clip, width=\textwidth]{images/06 - BIG_DATA.pdf}
\end{figure}

The coupon engine with personalized pricing is another aspect of
targeting. It provides customers with coupons based on their preferences
for their favorite products or for products that represent an upselling
opportunity. For instance, it may offer a coupon for a better quality
wine to encourage customers to switch to a higher quality product.

The personalized communication engine ensures that individual customer
preferences are taken into consideration when sending out mailings,
flyers, or pop-ups. Lastly, the instore proximity recommendation engine
provides customers with instant discounts or suggestions based on the
products they have already placed in their cart while shopping in a
supermarket.

As you can see, targeting is not limited to recommendations alone. It
encompasses a wide range of applications that utilize customer knowledge
and the same engine to enhance the overall customer experience.

\subsubsection{Personalization in Retail}

\begin{figure}[!h]
    \centering
    \includegraphics[page=90, trim = 0cm 3cm 1.5cm 4.5cm, clip, width=\textwidth]{images/06 - BIG_DATA.pdf}
\end{figure}

When considering personalized recommendations in retail, it's important
to recognize that not all companies can follow the same approach as
Amazon. While Amazon recommends widely purchased products, such as the
Amazon Choice, for items in your cart, this strategy may not be suitable
for all businesses. For example, if you are a supermarket competing on
quality, recommending the most frequently purchased product may not
align with your brand positioning.

By solely recommending mass-market products, even to customers who are
willing to purchase higher quality items, you risk negatively impacting
your business's bottom line. Non-personalized recommendations can lead
to reduced margins and revenue. Therefore, it's crucial to tailor
recommendations based on the specific needs and competitive position of
your company.

\begin{figure}[!h]
    \centering
    \includegraphics[page=91, trim = 0cm 0.7cm 1.5cm 1cm, clip, width=\textwidth]{images/06 - BIG_DATA.pdf}
\end{figure}

For instance, let's consider the example of upsell. If a customer
typically purchases the lowest-priced lemons, a personalized
recommendation engine should not suggest higher-priced organic lemons.
The price difference between the customer's usual choice and the organic
lemons is too significant. However, recommending Sorrento lemons, which
have a higher margin, could potentially generate greater profits, even
with a small discount. By offering a better product to the customer and
hoping they make the switch, you can increase your revenue.

\begin{figure}[!h]
    \centering
    \includegraphics[page=92, trim = 0cm 1.5cm 1.5cm 1cm, clip, width=\textwidth]{images/06 - BIG_DATA.pdf}
\end{figure}

Similarly, if a customer buys mass-market pasta, they would require
mass-market sauce. However, if they purchase premium or organic pasta,
it would be more appropriate to recommend a sauce with similar
characteristics. This level of attention to detail in personalized
recommendations can drive higher revenues and profits for your business.
By analyzing customer data and making these nuanced suggestions,
companies can maximize their earning potential.

\subsubsection{Architecture Considerations}

\begin{figure}[!h]
    \centering
    \includegraphics[page=97, trim = 0cm 2cm 0cm 4.5cm, clip, width=\textwidth]{images/06 - BIG_DATA.pdf}
\end{figure}

\begin{figure}[!h]
    \centering
    \includegraphics[page=98, trim = 0cm 9cm 1.5cm 4.8cm, clip, width=\textwidth]{images/06 - BIG_DATA.pdf}
\end{figure}

The solution provides measurable results, leading to increased sales and
customer satisfaction---a win-win situation for companies. Architecture
considerations are also important, especially when dealing with highly
personalized systems that require intensive computation. In such cases,
companies often generate large recommendation tables periodically, using
batch processing on commodity servers. These tables, which can be as
large as 300 gigabytes, are then accessed in real-time using
high-performance NoSQL key-value databases. This architecture allows for
efficient and effective processing of recommendations.

\subsubsection{Conclusions and Change Management}

\begin{figure}[!h]
    \centering
    \includegraphics[page=99, trim = 0cm 7cm 1.5cm 4.5cm, clip, width=\textwidth]{images/06 - BIG_DATA.pdf}
\end{figure}

In conclusion, the field of big data is continuously improving, with the
size of data increasing. To effectively manage this data, scalable and
cost-effective solutions are necessary. It is important to consider the
business benefits and find a case study where these benefits are
evident. Experimentation is crucial, as companies need to adopt an
innovative approach and be open to change. This requires a shift in
company culture, especially if the organization is accustomed to
standardization and control. Change management is essential and goes
beyond technology implementation. Specialized consultants can assist in
managing organizational change and facilitating the transition. However,
finding reliable change management consulting companies can be
challenging. Smaller consulting firms often offer valuable expertise in
this area. This concludes the discussion on big data and analytics.

\subsection{Closing Remarks}

As we reach the end of this course, I wanted to take a moment to provide
some closing remarks. This final class has been pre-recorded to allow
you additional time to study and adequately prepare for the upcoming
midterm. I hope that you have found this course informative and
engaging, and that it has provided you with valuable knowledge and
skills. I encourage you to continue your studies and apply what you have
learned in your future endeavors. Thank you for your participation, and
I wish you a pleasant evening.
