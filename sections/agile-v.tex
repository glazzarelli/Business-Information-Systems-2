\section{Agile - V Lecture}

\subsection{Microservices Architecture}

Alright, let's cover the final concepts before we wrap up the course.

\subsubsection{Importance for Managers and Developers}

In this section, we will discuss the importance of microservices
architecture for both managers and developers. Even if you are a manager
and not directly involved in development, understanding these concepts
is crucial because you may be responsible for making decisions about the
architecture based on factors such as budget and resources. We will also
touch upon the relevance of DevOps and development pipelines, time
permitting.

\subsubsection{Microservices vs Monolith}

Alternatively, you can rely on your technical team, who may recommend
implementing a microservices architecture. However, if your business
model is still evolving and your idea is not yet fully defined, a
microservices architecture may not be the best choice. In this case, it
might be more suitable to start with a quick and simple monolith
architecture. Once your business becomes more stable, your requirements
are well-defined, and your domain is established, you can transition
from a monolith architecture to a microservices architecture.

\subsubsection{Technical Debt and Team Dynamics}

\begin{figure}[!h]
  \centering
  \includegraphics[page=12, trim = 2.5cm 3.3cm 1cm 3.5cm, clip, width=\imagewidth]{images/09 - Bruna_microservices.pdf}
\end{figure}

During our discussion on the weaknesses of the monolithic architecture,
one crucial drawback stands out: the accumulation of technical debt. If
you're unfamiliar with the term, technical debt refers to the
consequences of having a complex application that becomes challenging to
modify or adapt to market changes. Over time, even simple changes
require significant effort. This issue typically arises after several
years of development.

However, it's worth noting that if you have a highly skilled and stable
development team working on a monolithic architecture, technical debt
can be minimized. But as your team grows and new members are added to
meet the need for speed and increased budget, the risk of accumulating
technical debt becomes more likely.

\subsubsection{Microservices Basics}

\begin{figure}[!h]
  \centering
  \includegraphics[page=19, trim = 1.8cm 5cm 1.5cm 5cm, clip, width=\imagewidth]{images/09 - Bruna_microservices.pdf}
\end{figure}

From a technical perspective, the answer to managing a larger and more
complex application is to adopt a microservices architecture. The
concept is straightforward: when faced with a large and
difficult-to-manage application, the solution is to break it down into
smaller, simpler units known as microservices.

\subsubsection{Granularity and Practical Examples}

While the term ``microservice'' may seem ambiguous at first, there are
no strict rules defining its size. Some argue that a microservice should
expose an API for a single entity or table, but I believe that level of
granularity is excessive. In my current project, the largest
microservice I have consists of around 12 tables.

\subsubsection{Microservices Philosophy}

\begin{figure}[!h]
  \centering
  \includegraphics[page=20, trim = 1.8cm 2.8cm 1.5cm 3cm, clip, width=\imagewidth]{images/09 - Bruna_microservices.pdf}
\end{figure}

While my project consists of around 400 to 500 tables, having only 12
tables in relation to the entire domain model is relatively small. The
philosophy behind microservices architecture is to focus on doing one
thing and doing it well.

\subsubsection{APIs and Inter-Service Communication}

In the context of microservices architecture, each microservice can be
viewed as an independent server-side application with its own database
(if required) and an API. The API serves as the sole means of
interacting with the business logic and data model within the
microservice.

\subsubsection{Data Duplication and Synchronization}

\begin{figure}[!h]
  \centering
  \includegraphics[page=21, trim = 1.8cm 3.5cm 1.5cm 3cm, clip, width=\imagewidth]{images/09 - Bruna_microservices.pdf}
\end{figure}

There is another possibility that I will explain later, but it is
crucial to avoid accessing or sharing databases between microservices.
This is a common mistake that can lead to a complex and disadvantageous
architecture. By coupling each microservice with a single data source,
you end up with the drawbacks of both microservices and monolithic
architectures. Instead, it is recommended to have different databases
based on the specific domain and requirements of each microservice. For
example, you can use a relational database or a NoSQL database, or even
a search engine, depending on the needs of the microservice.
Additionally, you have the flexibility to use different programming
languages within the same application. This means that if you have a
team where half is proficient in Java and the other half in Python, you
can implement your application using both technologies. While this may
not be the usual approach, it is feasible and can be beneficial for
certain projects.

Now, let's address the concept of data duplication in microservices,
which contradicts what you may have learned in database courses. In some
cases, duplicating data across microservices is necessary. For instance,
consider two microservices: one for sales and another for customer
support. Both microservices require access to the customer and product
tables. In this scenario, it is recommended to have separate instances
of these tables in each microservice.

\subsubsection{Event-Driven Architecture}

In a microservices architecture, it is common to have one microservice
designated as the model for changing and modifying entities, while the
other microservices only follow the master data and are not allowed to
make changes. As a result, the microservices that do not modify the data
typically only have the columns necessary for implementing their
specific functionality. This means that the data is duplicated across
multiple microservices.

To ensure that the duplicated data remains synchronized, you can use
various methods such as a message bus, a queue system, or a distributed
transactional log. These mechanisms facilitate the communication and
coordination between microservices, allowing them to exchange
information and keep their data consistent.

By designating one microservice as the master data source and making the
others read-only, conflicts can be avoided. However, there may be cases
where multiple microservices need to write to the same entity. In such
situations, it is crucial to ensure that they write to different columns
to minimize the risk of conflicts.

Overall, synchronizing duplicated entities in a distributed data model
requires careful consideration and the use of appropriate messaging and
coordination mechanisms to maintain data consistency across
microservices.

\subsubsection{APIs for Internal and External Clients}

Are you familiar with any open-source technologies? One example is
Kafka. Another is RabbitMQ. These technologies are used in the context
of microservices architecture, specifically for communication between
microservices.

Let's consider a scenario where we have two microservices: microservice
one and microservice two. Microservice one has its own table, called
table one, with an entity called T. When someone creates a new record
for this entity through the API, it is stored in the database. After the
record is created, we need to inform other microservices about this
event, even if we don't know if they are interested. To achieve this, we
use a message bus.

We publish the creation event, indicating that a new record has been
created. Microservice two, which also has its own version of the entity,
subscribes to this event. Whenever the event is published on the bus,
the bus notifies the subscribers, sending them the details of the newly
created record. Microservice two can then update or insert the new
record in its own model.

This pattern of communication is known as publish and subscribe.
Subscribers first subscribe to specific events, and then publishers
publish those events. This is how microservices can interact with each
other using events.

\begin{figure}[!h]
  \centering
  \includegraphics[page=22, trim = 1.8cm 4cm 1.5cm 3cm, clip, width=\imagewidth]{images/09 - Bruna_microservices.pdf}
\end{figure}

Another way for microservices to communicate is through APIs. APIs are
not only exposed to external clients, such as mobile or web
applications, but can also be used for communication between
microservices themselves.

\subsubsection{Strengths of Microservices}

\begin{figure}[!h]
  \centering
  \includegraphics[page=24, trim = 1.8cm 4cm 1.5cm 3cm, clip, width=\imagewidth]{images/09 - Bruna_microservices.pdf}
\end{figure}

Microservices architecture offers several strengths compared to other
paradigms. One of the key strengths is its reliability. In a
microservices architecture, even if a service is temporarily
unavailable, events are saved on a message bus and will eventually be
delivered. This ensures that messages are not lost and the system
remains robust.

Another strength of microservices architecture is its flexibility and
scalability. Unlike a monolithic architecture, where all modules write
and read from the same database table, microservices allow for
independent deployment and scaling of individual services. This means
that each service can be developed, deployed, and scaled independently,
providing greater flexibility and agility.

Additionally, microservices architecture enables better fault isolation.
Since each microservice is a separate entity, failures in one service do
not impact the entire system. This allows for easier troubleshooting and
maintenance, as issues can be isolated and resolved without affecting
the entire application.

From an architectural perspective, microservices architecture also
promotes the use of APIs. APIs serve as the primary means of interacting
with the business logic within each microservice. APIs can be
implemented using REST, which is a simple and widely used protocol for
sending and receiving JSON strings. This allows for easy integration
with various client applications, such as mobile apps or web
applications.

In terms of application development, microservices architecture supports
the implementation of different client applications, such as mobile apps
or web applications. These applications can consume the same API
provided by the microservice, eliminating the need for duplicating
functionality. This promotes code reuse and reduces development effort.

Overall, the strengths of microservices architecture lie in its
reliability, flexibility, scalability, fault isolation, and support for
API-driven development. These strengths make it a powerful and popular
choice for building modern, distributed systems.

\subsubsection{Autonomy and Evolution}

In addition to exploring the strengths of microservices applications, it
is important to acknowledge their weaknesses. While microservices offer
significant advantages, there are also drawbacks that need to be
considered. One of the key strengths of microservices is autonomy, where
the business logic and data model are combined.

\subsubsection{Scalability and Team Structure}

One of the key factors contributing to their success is their approach
to architecture. In the past, architectures like web services with
components such as enterprise service bus and business process server
failed because they separated the business logic from the data model.
This resulted in the business logic being represented as XML, trying to
orchestrate calls between different applications through a business
process server, while the enterprise service bus attempted to facilitate
communication between endpoints. However, this approach led to a
distributed data model and a disconnect between the business logic and
the data model.

To ensure the success of microservices, it is crucial to keep the
business logic as close as possible to the data model. The business
logic and the API should be treated as atomic entities. The API serves
as the contract through which functionality is exposed, while the
implementation details are hidden. By exposing only minimal
implementation details, you have the flexibility to make changes to the
implementation without affecting the API. This encapsulation of
implementation details allows for independent evolution of
microservices.

Another advantage of microservices is the ease of problem determination.
In a monolithic architecture with millions of lines of code, it can be
challenging to identify the source of a malfunction. However, with
microservices, each microservice has its own log file, and there are
tools available to collect and search through these log files. This
makes troubleshooting and debugging much more manageable.

Scalability is another strength of microservices. For example, during
peak seasons like Christmas, when there is a high demand for a product,
microservices can be deployed on the cloud to handle the increased
workload. This scalability ensures that the system can handle the surge
in traffic without compromising performance.

\subsubsection{Cloud Readiness and Hybrid Solutions}

Let's consider the example of the catalog microservice, the order
microservice, and the user customer. The catalog microservice is the
most impacted by user navigation on your website. Normally, you may have
one or two instances of this microservice in the cloud throughout the
year. However, during the Christmas period, you may need to scale up and
have multiple instances of the catalog microservice. On the other hand,
the order microservice is fine with just three instances. This selective
scaling allows you to address the scalability issue by only scaling the
specific functionality that requires it. In contrast, with a monolithic
architecture, you would need to scale the entire monolith, resulting in
higher costs for expensive virtual machines on platforms like Amazon Web
Services.

Another significant advantage of microservices is the ability to divide
development efforts among multiple teams. In the project I am currently
working on, we have a team of 40 people. It would be impossible for all
40 people to work on the same monolithic application. Instead, we divide
the team into four smaller teams, each consisting of 10 developers, two
front-end developers, and one database specialist. The database
specialist rotates among the teams, so each team does not have a
dedicated specialist. Each team takes ownership of a specific
microservice, treating it as an internal product. This distribution of
people and their attachment to different microservices allows for
scalability. However, it can also create challenges if the individuals
assigned to a particular microservice are unavailable. To avoid
dependency on specific developers, it is advisable to rotate team
members and not assign exclusive ownership of a microservice.
Nevertheless, in situations where speed is crucial within a short
timeframe, temporary ownership can be a viable solution.

For example, Spotify was developed by a collection of tribes, each
consisting of nine people. The application is composed of different
microservices, with each microservice managed by a different tribe. Each
microservice is treated as a separate product internally. The catalog,
suggestion engine, and billing engine are examples of components managed
by different tribes and teams. This approach allowed Spotify to build
its platform in a relatively short period.

Another advantage of microservices is their readiness for cloud
environments. In large companies, hybrid solutions are often used,
combining resources from private clouds and internal servers.
Microservices, with their use of networking protocols for communication,
are well-suited for such hybrid cloud setups.

\subsubsection{Drawbacks and Considerations}

In a microservices architecture, it is common to have a combination of
on-premise servers and cloud providers. This allows for flexibility and
scalability, as you can scale specific microservices on the cloud during
peak periods, such as Christmas, to ensure a smooth user experience.
This level of scalability is not possible with a monolith running on
on-premise servers. The advantages of this approach are evident, but
it's important to consider potential drawbacks and points of attention,
such as increased complexity.

\subsubsection{Distributed Transactions and Saga Pattern}

In a microservices architecture, it's important to consider the specific
needs and complexity of your application. If you have a simple startup
with a straightforward idea, it may be more beneficial to start with a
simple monolith architecture rather than investing heavily in
technology. This allows you to focus on developing functionalities
without the added complexity of microservices.

However, if you have a Formula One team and need to compete in the
championship, a microservices architecture is necessary. It requires a
skilled team to manage the dynamic nature of bounded contexts. It's
common for the requirements and domain of a microservice to change
during the project, resulting in the need to move entities between
microservices. This can be a costly process as it requires deep code
refactoring.

One important consideration in microservices architecture is managing
transactions. In a monolith, transactions are straightforward as they
operate within a single database. You can start a transaction, perform
multiple operations, and commit the changes to the database. If
something goes wrong, you can roll back the transaction, ensuring that
no changes are persisted.

However, in microservices, managing transactions becomes more complex.
If a transaction involves entities from multiple microservices, ensuring
that all changes are either written or none are written becomes
challenging. The distributed nature of microservices means that you
can't rely on a single database to manage transactions across services.
This requires careful design and implementation to ensure data
consistency and reliability.

To enable rollback functionality in a microservices architecture, the
Saga Pattern can be utilized. This design pattern facilitates
distributed transactions among microservices, ensuring that the database
remains atomic, consistent, and durable. However, it is important to
note that sagas cannot achieve isolation, which means that other
microservices can observe the ongoing transaction while it is being
orchestrated by an orchestrator using messaging.

In contrast, isolated transactions only reveal the transaction's results
to the rest of the world once it has been committed. By default,
databases allow others to see the work being performed, but it is
generally preferred to display the transaction's outcome only after it
has been committed. With the saga pattern, this level of isolation is
not possible. Instead, sagas operate on an ``everything or nothing''
basis. If an error occurs during the execution of a transaction, the
saga will initiate a rollback, triggering the rollback command in each
microservice involved in the distributed transaction. This ensures that
any work contributed by the microservices is deleted.

In summary, it is important to understand the concepts of ``publish and
subscribe,'' microservices, their strengths, and when to use them.
Additionally, it is crucial to be aware of the challenges associated
with distributed transactions.

\subsection{DevOps and Development Pipelines}

\subsubsection{Introduction to DevOps}

DevOps is a crucial aspect of modern software development. It focuses on
collaboration and communication between development teams and operations
teams to ensure efficient and reliable software delivery. The goal of
DevOps is to automate processes, improve deployment frequency, and
enhance the overall quality of software.

By adopting DevOps practices, organizations can achieve faster
time-to-market, increased productivity, and improved customer
satisfaction. DevOps emphasizes the use of automation tools, such as
continuous integration and continuous deployment (CI/CD), to streamline
the software development lifecycle.

In traditional software development, there is often a disconnect between
development and operations teams, leading to delays, errors, and
inefficiencies. DevOps aims to bridge this gap by fostering a culture of
collaboration and shared responsibility. Development teams work closely
with operations teams to ensure that software is developed, tested, and
deployed smoothly.

One of the key principles of DevOps is the concept of ``infrastructure
as code.'' This means that infrastructure, including servers, networks,
and databases, is managed and provisioned through code, allowing for
greater scalability, flexibility, and reproducibility.

Another important aspect of DevOps is continuous monitoring and
feedback. By monitoring the performance and usage of software in
real-time, teams can identify and address issues promptly, ensuring a
seamless user experience.

In summary, DevOps is a transformative approach to software development
that emphasizes collaboration, automation, and continuous improvement.
By adopting DevOps practices, organizations can achieve faster, more
reliable software delivery, and ultimately, better meet the needs of
their customers.

\subsubsection{Project Management and Agile Methodologies}

Now, let's discuss the first
direction, which focuses on project management and how to effectively
organize a successful team. This is known as the project management
axis.

In the realm of DevOps and development pipelines, it is highly
recommended to adopt agile methodologies. One popular option is Scrum,
but there are other agile principles that can be followed as well. Agile
methodologies are particularly relevant for certain career paths, such
as the Scrum Master role. While technical knowledge is not a requirement
for this role, having a solid understanding of technology can be
beneficial. Additionally, there are other important roles within agile
teams, including the product owner. If you have a clear vision of how
the product should be defined, you can take on the role of the product
owner. The second aspect to consider is the technology axis.

\subsubsection{DevOps Principles and Practices}

In the realm of architectural paradigms, one option to consider is the
microservice architecture. This approach is particularly suitable if you
prioritize agility and speed. However, it's important to note that the
monolith architecture can also be a viable choice. Regardless of the
architecture you choose, implementing DevOps practices is crucial.

Now, let's delve into what DevOps actually means. It's a term that
combines development and operations, and it has gained significant buzz
in recent years.

\subsubsection{Cultural Shift and Tooling}

\begin{figure}[!h]
  \centering
  \includegraphics[page=4, trim = 1.8cm 3.5cm 1.5cm 3cm, clip, width=\imagewidth]{images/10 - Bruna_devops.pdf}
\end{figure}

In the past, development and operations teams worked separately, with
different priorities and objectives. The development team focused on
innovation, speed, and flexibility, while the operations team
prioritized reliability, security, and problem management. This created
a divide between the two teams, both organizationally and in terms of
their goals.

The development team worked efficiently, following agile practices and
producing software at a fast pace. However, when it came to deploying
the software in a production environment, they had to rely on the
traditional operations team. This process involved opening a ticket and
waiting for the operations team to handle it, which often took a long
time. This delay meant that the software was not being utilized and
valuable feedback was not being received.

The operations team had their own challenges, such as dealing with
emergencies and interruptions. They also had a different way of working,
requiring issues to be raised through an issue tracker. This created a
barrier between the development and operations teams, causing delays and
miscommunication.

To address these issues, the concept of DevOps emerged. DevOps aimed to
break down the wall between development and operations by implementing
new methodologies and tools. While it took 20 years for agile practices
to become widely adopted, DevOps gained traction more quickly due to its
focus on cultural change and the use of tools to facilitate the
transition.

By embracing DevOps, organizations could bridge the gap between
development and operations, enabling faster and more efficient software
deployment. This cultural shift and tooling allowed for better
collaboration, improved communication, and the ability to deliver
software that met customer needs in a timely manner.

\subsubsection{DevOps Lifecycle and Automation}

\begin{figure}[!h]
  \centering
  \includegraphics[page=7, trim = 1.8cm 3.5cm 1.5cm 3cm, clip, width=\imagewidth]{images/10 - Bruna_devops.pdf}
\end{figure}

The concept of DevOps is a combination of patterns aimed at enhancing
collaboration between development and operations. It involves viewing
the entire software lifecycle management as a unified process, rather
than separate development and operations processes. Another variation of
DevOps is DevSecOps, which emphasizes the importance of integrating
security into the process.

\begin{figure}[!h]
  \centering
  \includegraphics[page=8, trim = 1.8cm 6cm 1.5cm 3cm, clip, width=\imagewidth]{images/10 - Bruna_devops.pdf}
\end{figure}

The goals of DevOps are to increase speed, improve reliability, and
reduce costs. There are three key ways to implement DevOps. The first is
to adopt a systems thinking approach, recognizing that the goal is not
just to develop an application, but to ensure its successful use by
clients or customers. The second way is to continuously improve the
process by seeking feedback and monitoring performance. This allows for
experimentation and the exploration of new ways to achieve better
results.

The symbol of DevOps is often represented by an infinity symbol, which
signifies the different phases of the software lifecycle process.

\subsubsection{Roles and Continuous Improvement}

In the DevOps cycle, you start by collecting requirements and planning
your iteration. Then you move on to coding and building the software.
During this process, there are various steps, including unit testing,
manual testing, and automatic user acceptance testing. Once the software
is ready, it is released and stored in an archive with a version number.
From there, it is deployed by other software. Customers and users then
start using the software, and it is important to monitor it for any
issues or resource constraints. The next step is to plan and code the
next set of important features.

\begin{figure}[!h]
  \centering
  \includegraphics[page=11, trim = 1.8cm 3.7cm 1.5cm 3cm, clip, width=\imagewidth]{images/10 - Bruna_devops.pdf}
\end{figure}

To implement DevOps, you can utilize a range of tools and technologies.
For coding, you can use a code repository like Git, and development
environments such as Eclipse, Visual Studio Code, or IntelliJ. To build
the software, automation tools like Maven, Apache Ant, or Jenkins can be
used. Testing can be automated using tools like JUnit. The orange and
yellow sections represent the operations and development aspects,
respectively. In the past, these were handled by separate teams, but now
they are part of the same process. Operations teams can set up and
implement toolchains to provide developers with self-service options for
releasing and deploying software.

\begin{figure}[!h]
  \centering
  \includegraphics[page=12, trim = 1.8cm 5.5cm 1.5cm 3cm, clip, width=\imagewidth]{images/10 - Bruna_devops.pdf}
\end{figure}

\begin{figure}[!h]
  \centering
  \includegraphics[page=13, trim = 1.8cm 2.5cm 1.5cm 3cm, clip, width=\imagewidth]{images/10 - Bruna_devops.pdf}
\end{figure}

The main goal of DevOps is to automate as much manual work as possible
in bringing software from a code repository to a production environment.
The ability to push a single button and deploy a complex solution
composed of multiple microservices is a significant achievement.
Automation makes the process repeatable and allows for frequent
production deployments. By optimizing the overall flow of the process,
DevOps enables the management of complex distributed architectures. The
focus is on optimizing the entire process rather than micro-optimizing
individual areas like tests or code. In the DevOps approach, different
roles are required to participate in the process, but now these roles
should be part of a single team rather than separate teams.

\subsubsection{Final Thoughts and Summary}

\begin{figure}[!h]
  \centering
  \includegraphics[page=14, trim = 1.8cm 3.5cm 1.5cm 3cm, clip, width=\imagewidth]{images/10 - Bruna_devops.pdf}
\end{figure}

Another important principle in DevOps is to address problems in the
process as soon as they arise, rather than allowing them to progress to
the next step. For example, if you are implementing a feature and you're
unsure about a particular algorithm, it's better to solve the problem at
that stage rather than relying on the unit test team to catch it later.
By resolving issues at each step, you ensure that the subsequent steps
receive good inputs and can produce the desired outputs.

While there are other roles that can be part of DevOps, they are not the
focus here. The key is to promote continuous improvement. If you
currently release to production once a month, consider aiming for a
weekly release. In some projects, we even achieved daily production
releases. This approach is more relaxing because if a mistake is made,
it can be fixed the next day. It also allows for quicker response to
customer feedback on new functionalities, rather than waiting for months
to make corrections.

\begin{figure}[!h]
  \centering
  \includegraphics[page=16, trim = 1.8cm 5.5cm 1.5cm 3cm, clip, width=\imagewidth]{images/10 - Bruna_devops.pdf}
\end{figure}

By increasing the frequency of releases and gaining experience in
managing the entire software lifecycle, you become more efficient. This
repetition and learning process is valuable. In summary, these
principles of addressing problems early, promoting continuous
improvement, and increasing release frequency contribute to a more
efficient and effective DevOps approach. If you have any questions about
these concepts, please feel free to ask.

\subsection{Career Advice and Industry Insights}

\subsubsection{Software Quality and Enterprise Architecture}

Agile methodologies are the future of project management. This is a
crucial message. In the past, there was a misconception that software
implementation could be outsourced to any team around the world because
it was considered a simple commodity, like buying oil or electricity.
However, big companies have come to realize that this approach was a
mistake. By outsourcing software development, they lost control over
their own technology and decision-making. The person working directly
with the technology, who understands the problem and the domain, should
be the one making important decisions. Smart companies have started to
bring knowledge back in-house by hiring enterprise architects and
building internal developer teams. They may still work with external
software providers and system integrators, but they also ensure that
their own developers are involved in every project to learn and transfer
knowledge. The quality of software is paramount because high-quality
software allows for faster implementation of new functionalities at a
lower cost. On the other hand, quick and dirty solutions lead to
technical debt and eventually hit a wall of sustainability. When this
happens, it becomes more expensive to re-implement everything rather
than making changes to existing software. Therefore, software quality is
crucial, and it is important to have a clear understanding of the design
patterns used at all levels, from individual components to the entire
enterprise architecture.

\subsubsection{Choosing the Right Project and Company Size}

I have worked for companies that faced significant challenges due to
poor data architecture and lack of enterprise-level planning. One
particular company had 800 applications, each with its own database, and
a complex network of scripts exchanging data between them. This resulted
in spikes of CPU usage and a loss of control over the architecture. When
they needed to add new functionality, they kept adding layers of
complexity without considering the bigger picture. Eventually, they had
to halt their business operations for several days, resulting in
significant financial losses.

This experience taught me three important lessons. First, the
architecture and software design are crucial. Without proper planning
and oversight, a company can quickly lose control of its systems.
Second, it's important to avoid being a ``monkey developer'' who relies
on manual, repetitive steps for tasks like deployment. Taking the time
to automate these processes can save time and effort in the long run.
Lastly, when choosing a project or company to work for, it's important
to consider the size and complexity of the organization. While big
companies offer stability and resources, smaller companies may provide
more opportunities for growth and a closer-knit team environment.

I've seen colleagues and young entrepreneurs who either overcomplicated
their architectures or created overly simplistic solutions. Both
approaches can lead to technical debt and hinder future development.
It's important to strike a balance and be adaptable to changing needs.
While there is no set formula for when to switch to a microservice
architecture, it's crucial to recognize when the current system is
becoming too complex to handle future requirements.

In some cases, a temporary shift to a command-and-control management
style can be effective for achieving short-term goals. However, this
approach is not sustainable in the long run and may lead to team members
leaving. It's important to create an environment that fosters
collaboration and allows individuals to work in a way that suits their
preferences.

In today's competitive job market, good developers are highly sought
after. Recruiters constantly reach out with enticing opportunities,
making it challenging to choose the right path. While working for a big
company may offer stability and benefits, some individuals prefer a
smaller, more intimate work environment where they feel valued and part
of a close-knit team. Ultimately, it's important to weigh the pros and
cons and choose a path that aligns with your goals and values. Starting
with a smaller company can provide valuable experiences and exposure to
a wider range of tasks and responsibilities.

When starting your career in a large company, it's common to be assigned
to a specific role within a narrow sector. As a junior developer, you
may find yourself working on the same task for many years. This is
because larger companies tend to have more specialized roles and may not
immediately place you in a higher-level position like an architect. If
you want to gain exposure to different aspects of a project, it may be
beneficial to start your career in a smaller company. Once you have
gained more experience, you can then transition to a larger company.

For example, in my own experience, I worked for a company with 150
employees.

In 2000, the company was founded, and I joined in 2003. I am still
working there to this day. In January of last year, we made a strategic
decision to focus on software quality in order to compete in the market.
Our approach was to offer higher quality software, even if it meant
being more expensive. As a result of our success, we were acquired by a
German company called Adesso. They recognized the value of our
quality-focused mindset and are now using us as their first Italian
company to enter the Italian market. I hope this explanation has been
helpful in understanding our journey.
