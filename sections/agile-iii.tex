\section{Agile - III Lecture}

\subsection{Introduction to Agile and Scrum}

Good morning! Let's take a moment to discuss the Agile methodology,
specifically the Scrum framework.

\subsection{Scrum Framework Components}

\subsubsection{Roles in Scrum}

Understanding the concepts of Agile and Scrum is relatively simple, but
the true value of these methodologies becomes apparent when working with
a team. To fully appreciate the benefits of the Agile and Scrum methodology, it
is recommended to have a team of at least seven people. This number
allows for effective meetings and clear product management. Let's
summarize the key elements again: three roles, five events, and three
artifacts.

The three roles in Scrum are the Product Owner, Scrum Master, and
Developer. The term ``Developer'' encompasses not only software
developers but also any individuals who contribute their skills to the
project, such as graphic designers.

\subsubsection{Scrum Events}

In the Scrum framework, there are several important events that help
facilitate effective collaboration and progress. The three key events
are Sprint Planning, Daily Scrum, and Sprint Review. Additionally, there
are two other important events: Backlog Refinement and Retrospective.

Sprint Planning is a crucial event where the Scrum Team plans the work
to be done in the upcoming Sprint. It involves selecting the items from
the Product Backlog that will be worked on and creating a Sprint Goal.

The Daily Scrum is a short daily meeting where the Scrum Team
synchronizes their work. Each team member provides an update on what
they have accomplished since the last meeting, what they plan to do
next, and any obstacles they are facing.

The Sprint Review is held at the end of each Sprint and provides an
opportunity for the Scrum Team to showcase the work they have completed.
Stakeholders are invited to provide feedback and suggest any changes or
improvements.

In addition to these three events, there are two other important events
in the Scrum framework. Backlog Refinement is a collaborative session
where the Scrum Team reviews and updates the Product Backlog. This helps
ensure that the backlog is well-prepared and prioritized for future
Sprints.

The Retrospective is a valuable event that occurs at the end of each
Sprint. The Scrum Team reflects on their work and identifies areas for
improvement. This event allows the team to continuously enhance their
processes and increase their effectiveness.

By understanding and effectively utilizing these events, teams can
maximize their productivity and deliver high-quality results in an
iterative and incremental manner.

\subsubsection{Collaboration and Communication}

The reason why it is important for the methodology to make all these
events mandatory is because they are valuable time investments in the
project. Without a methodology like this, when you fall behind schedule,
you tend to isolate yourself in front of your computer, trying to catch
up on the accumulated delay. However, even if you are behind, these
moments are crucial. As someone who enjoys working independently, I
understand the temptation to skip these events. But coordinating with
others is essential. Although they may seem like additional overhead
because you are not directly developing software, it is important not to
waste too much time and instead invest the right amount of time in these
time-boxed events.

For example, in the daily stand-up meetings, it is important for a team
of seven to nine people to keep it to 15 minutes. This way, you can
quickly share what you accomplished yesterday, what you plan to do
today, and if you have any problems. After the meeting, you can schedule
additional discussions with team members to delve into further details.
These events serve as the rules and guidelines for effective
collaboration and communication within the methodology.

\subsection{Product Backlog and User Stories}

\subsubsection{Product Backlog Management}

Let's begin with the primary artifact: the product backlog. The product
backlog is a comprehensive list of tasks and requirements that need to
be completed in order to finish the project.

\subsubsection{Prioritization and Readiness}

\begin{figure}[!h]
  \centering
  \includegraphics[page=31, trim = 0.5cm 4.2cm 2cm 2.7cm, clip, width=\imagewidth]{images/08 - Bruna_scrum.pdf}
\end{figure}

The product owner is accountable for the artifact known as the product
backlog. It is their responsibility to maintain and understand the items
in the backlog, as well as explain them to the development team. During
sprint planning, the product owner can ask questions about the items to
ensure clarity. However, it may not be possible to analyze every item in
detail during sprint planning. Therefore, it is crucial to prioritize
the product backlog. The most important tasks should be at the top,
while less important ones are lower down. In Agile methodology, it is
important to avoid wasting time on detailed analysis for lower priority
items. Instead, focus on making the top items ready for development.

\subsubsection{User Stories and Detail Levels}

When working with a product backlog, it's important to strike a balance
between analyzing technical and functional details. Spending too much
time on detailed analysis for items that may not be chosen for several
months can be wasteful, as things can change in the meantime. This
concept is known as ``energy.''

\begin{figure}[!h]
  \centering
  \includegraphics[page=32, trim = 0.5cm 6.5cm 2cm 2.7cm, clip, width=\imagewidth]{images/08 - Bruna_scrum.pdf}
\end{figure}

The product backlog typically consists of user stories, which are
concise descriptions of business requirements. These stories can be
functional, describing a specific functionality, or
technical/non-functional, related to the technical aspects necessary for
the development of functional stories. They can also include stories for
bug fixes or fine-tuning. User stories focus on explaining the
requirement without delving into the solution. They provide a sufficient
level of detail to be actionable, but not so much that they become
unrealistic. Striking this balance can be challenging, but it is crucial
for effective backlog management.

\subsection{Sprint Planning and Execution}

\subsubsection{Sprint Planning and Task Breakdown}

Sometimes, user stories lack sufficient detail to guide developers on
what needs to be implemented. On the other hand, providing too much
information can encroach on the team's responsibility to come up with
the solution. The key is to strike a balance. During sprint planning,
which typically lasts two to three weeks, the team selects items from
the top and adds them to the sprint backlog. Each item is then discussed
in detail. To better understand the complexity of a user story,
developers can break it down into sub-tasks. Ideally, every team member
can contribute to breaking down user stories into multiple tasks.
However, this activity can be time-consuming, especially for larger
projects with teams of around nine people. In such cases, sprint
planning sessions can last up to two days.

\subsubsection{Sprint Execution and Testing}

The task explosion in sprint planning and execution can be a challenging
process. It is not always straightforward and can vary in difficulty.
One of the reasons for this is the lack of understanding of the tasks at
hand. Sometimes, technical details are needed to fully comprehend how to
implement them. Additionally, during sprint planning, there may be
missing information that is necessary for task execution.

In traditional development, the process follows a sequential order:
analyze, implement, and then test. However, with the agile methodology,
such as the Scrum framework, the process is different. Here, you
analyze, implement, and then test only a portion of the functionality.
This approach serves as a contract between the product owner and the
development team, defining the concept of ``done.''

\subsubsection{Adapting and Improving}

\begin{figure}[!h]
  \centering
  \includegraphics[page=33, trim = 0.5cm 4.2cm 0.5cm 2.7cm, clip, width=\imagewidth]{images/08 - Bruna_scrum.pdf}
\end{figure}

During the sprint planning phase, it is crucial to collaborate with your
product owner to define the ``definition of done'' for the project. This
definition typically includes criteria such as the software working in
the integration environment, a functional demo during the sprint review,
and the development of unit tests. It is important to showcase the
project's progress to the customer after each sprint to receive valuable
feedback. Based on this feedback, you may need to make adjustments to
the backlog, such as changing priorities, adding or removing items.

Another important meeting in the process is backlog refinement. This
meeting allows for the discovery of missing or unnecessary functionality
during implementation. Customer participation in the review often leads
to the realization that certain requested features are not needed or
were incorrect. During this meeting, you can add new requests, change
priorities, and remove items from the project scope.

Sometimes, delays may occur, and instead of changing the deadline, it
may be necessary to reduce the scope of the first release. By
implementing the most important features first, you can still achieve
your objectives. Less critical functionality can be removed and included
in a second release.

Lastly, the retrospective is a crucial moment for reflection. Even if
you are behind schedule, it is recommended to invest at least one hour
after the sprint review to identify actions that can improve future
performance.

\subsection{Estimation and Velocity}

\subsubsection{Poker Card Estimation}

For example, if you notice a lack of communication between the front-end
application developers and the API developers, it may be beneficial to
create separate rooms for each user story. This allows the people
involved in that specific user story to work together, either physically
or remotely, throughout the day. By doing this, you can improve
communication and collaboration between team members. It's important to
remember that each project is unique, so you may need to adapt your
strategies to fit the specific needs of the project.

Now, let's recap what we've covered so far. If you have any questions or
if you've used this process with your colleagues, feel free to ask. To
start the project, you gather the requirements and create a product
backlog. This can be done using a simple tool like an Excel file. The
entire team is responsible for building and maintaining this artifact,
while the product owner is accountable for its organization and
explanation. The product backlog should be accessible to all team
members.

Next, you organize the first sprint by conducting a sprint planning
session with the entire team. During this session, you define the sprint
backlog, which should only include the most important tasks. To estimate
the effort required for each task, you can use a technique called poker
card estimation. Each team member has a set of cards with numbers from
the Fibonacci series, which they use to estimate the complexity of each
task.

\subsubsection{Understanding Team Velocity}

In order to estimate the complexity of user stories, each team member is
given a deck of cards. The product owner explains the requirements of
the user story and asks the team to assess the complexity. Each team
member selects a card and places it face down on the table. The cards
are then revealed simultaneously to see if the estimations are similar.
The numbers on the cards correspond to the Fibonacci series and can
indicate the number of days needed to implement the functionality or
simply measure the complexity.

These estimations are then translated into a single unit of measure
called function points. At the start of the project, one function point
is equivalent to one day of work. However, this relationship can change
throughout the project. As the team improves their skills and
capabilities, the value of one function point may increase, such as
becoming 2.5. This relationship is known as velocity.

Velocity represents the team's speed and can be graphed over different
iterations to track their improvement or any changes in their pace.

\subsection{Practical Exercise}

\subsubsection{Setting Up a Product Backlog}

\begin{figure}[!h]
  \centering
  \includegraphics[page=35, trim = 0.5cm 4.2cm 1.5cm 2.7cm, clip, width=\imagewidth]{images/08 - Bruna_scrum.pdf}
\end{figure}

Now, let's imagine that each team member expresses their estimation of
complexity for a particular task. Since the estimations are quite
similar, you associate this level of complexity to your story.
Typically, you would choose the highest number as the estimation. So,
let's discard the three and consider it as the complexity. For the first
item in the backlog, we estimate that it will take three days based on
our starting velocity.

However, during the poker card estimation, one team member puts an eight
while others put a three. This is a moment to start a discussion and
understand why there is a difference. This is crucial because that
person might have knowledge or insights that others don't. For example,
they might know about a technical difficulty or a prerequisite for
implementing the user story. After the discussion, we re-estimate using
the poker card method. Usually, when the developer's explanation is
reasonable, others adjust their estimations accordingly. In this case,
we discover that the second item can be estimated with eight story
points. Story points are another term for function points, but we use it
because we are dealing with user stories.

It's important to note that the relationship between story points and
the time needed to implement them can change throughout the project.
This is why the poker card estimation technique is valuable. By asking
for individual estimations, we prevent senior members from influencing
the estimations of junior members. This is crucial because even junior
members can bring value to the team. They might have previous project
experience or knowledge that senior members don't have in this specific
context. Everyone's input is valuable.

There are two special cards in the estimation process. One card
indicates that the person cannot estimate because they lack knowledge or
information. This is considered an infinite estimation. The other card,
symbolized by a coffee cup, indicates that the person is requesting a
break.

Now, let's review some rules. In the team, we have the product owner who
owns the product and decides what needs to be implemented. The Scrum
Master, on the other hand, is not a master of anything. They are usually
considered a servant leader. They have a strong personality and
excellent communication skills. Their main role is to ensure the
methodology works smoothly. They organize the calendar for different
meetings and their technical skills may not be the primary focus.
Successful Scrum Masters can come from various backgrounds, such as
philosophy, as long as they can effectively interact with people and
create a comfortable project environment.

Now, let's put this into practice with a small exercise to simulate the
creation of a product backlog.

\subsubsection{Organizing the First Sprint}

Let's begin by simulating the organization of the first sprint and
creating a plan for it. Here is the project description:

The CEO, John Smith, wants to throw a fun and informal party for his
employees, directors, and board members at his house. The main objective
of the project is to ensure that the party is enjoyable. If the weather
permits, the CEO would prefer to spend more time in the garden than
inside the house. A buffet is required to cater to approximately 100
people, and a barbecue is also desired. The CEO wants to have music
during the event, so the band or a DJ needs to be hired. Additionally,
before the party starts, there will be a 30-minute company meeting where
the CEO will present the company's current status.

These are the requirements for the project. The idea is to divide
yourselves into four or five teams. You can organize yourselves in the
same way you are currently organized, for example, as teams one, two,
three, four, and five.

\subsubsection{Event Planning Exercise}

You can access the team board that has been prepared for this exercise.
The address is provided, and if you need it, I can also send it to you
via email. The board contains the team conversation, and as the team
lead, you are responsible for overseeing the process. You are also the
product owner, meaning you can ask me for further explanations about the
requirements.

To start, go to the top left of the board where you will find the empty
product backlog. Your task is to write a small yellow post-it note for
each item that you want to include in the backlog. These items should
represent the characteristics of the stories, but keep in mind that this
exercise does not involve implementing software. The objective is to
achieve the goals set for the event planning exercise.

Organizing an event for 100 people may not be easy, especially if you
have never done it before. However, you have the opportunity to
prioritize the items on the backlog. If you need a better understanding
of the priorities, you can ask me for further explanation as your
product owner. Your job is to write down the tasks required to organize
the event.

Now, let's divide into teams. Number one, you can start. Number three,
you have decided to be number three. Let's continue with number two,
then number six, and so on. This is a self-organizing team, so please
begin. You have 10 to 15 minutes for this important task. If you need
cards, you can use the ones provided. If you have any trouble accessing
the board, let me know, and I will assist you. You can also refer to
what your colleagues did last year for inspiration. If you need
suggestions, feel free to ask. Some examples include checking weather
forecasts, creating a backup plan for rain, setting a budget, cleaning
the venue, arranging for food and drinks, selecting bedrooms, mapping
accessible areas, planning for after-party cleanup, arranging for
catering, hiring a DJ and bartender, renting equipment like projectors,
tables, and chairs, sending out invitations, and organizing a wardrobe
room.

\subsection{Conclusion and Questions}

As the product owner, I understand that you have many questions and
requests regarding the project. Please feel free to ask me anything you
need clarification on or any concerns you may have. I am here to assist
you.