\section{Knowledge Management - II Lecture}

This is the continuation of the
recording on knowledge management.

\subsection{Behavioral Strategy for Knowledge
  Management}\label{behavioral-strategy-for-knowledge-management}

\subsubsection{Knowledge Sharing Culture}\label{knowledge-sharing-culture}

Let's discuss the behavioral strategy for managing knowledge and
creating a knowledge sharing culture. Companies have the power to shape
their environment, both physically and culturally, to encourage
knowledge sharing. One effective way to do this is by implementing
shared spaces and open office layouts. These spaces remove barriers
between individuals, allowing for better collaboration and
communication.

Open office spaces are particularly interesting because they provide an
opportunity to observe and learn from others. By working in close
proximity to colleagues, you can see what they are working on and hear
their conversations. This allows you to build a mental map of who
possesses specific knowledge and who is working on certain projects.
Having this knowledge is invaluable because it helps you identify the
right people to approach for input or assistance on specific issues.

\subsubsection{Legitimizing Knowledge
  Sharing}\label{legitimizing-knowledge-sharing}

Legitimizing knowledge sharing is a crucial aspect of knowledge
management, especially in hierarchical organizations where sharing
knowledge is often restricted. In such organizations, knowledge cannot
be shared without the authorization of the person in a higher position
or the one responsible for the specific unit. For instance, if someone
from a different unit needs a report that you possess, you are not
allowed to share it without permission.

To promote knowledge sharing in practice, it is essential to legitimize
and change these strict hierarchical rules, at least for certain
categories of knowledge. It is crucial for people to understand the
rules and for the rules to clearly define what types of knowledge can be
shared and what cannot. By establishing clear guidelines, organizations
can encourage a culture of knowledge sharing and facilitate the flow of
information among employees.

\subsubsection{Knowledge Management
  Systems}\label{knowledge-management-systems}

\begin{figure}[!h]
  \centering
  \includegraphics[page=15, trim = 1.5cm 7cm 1.5cm 4cm, clip, width=\imagewidth]{images/05 - KM.pdf}
\end{figure}

Knowledge management systems play a crucial role in facilitating
knowledge sharing within organizations. These systems provide the
necessary support for effective knowledge sharing. The definition of
knowledge management systems is quite broad, encompassing various
components of an Enterprise Resource Planning (ERP) system. In fact,
almost all modules of an ERP can be classified as a knowledge management
system or can be utilized as such.

\begin{figure}[!h]
  \centering
  \includegraphics[page=16, trim = 3cm 3.8cm 4cm 5.3cm, clip, width=\imagewidth]{images/05 - KM.pdf}
\end{figure}

The chart provided represents the modules of an organization's
information system from a knowledge management perspective. Many of
these modules align with the ERPs we have discussed previously. We may
delve deeper into some of these modules in future discussions. The terms
``executive,'' ``operational,'' and ``analytical'' are familiar to us
from our classification of ERP functionalities in BIS 1 and the early
stages of BIS 2. The breadth of knowledge management systems can be
somewhat disappointing from a technical standpoint, as we typically
expect specific technologies to support particular applications.
However, in the case of knowledge management, the scope is extensive.
Even the definition of knowledge itself is broad. Therefore, any module
within an ERP can be considered a knowledge management system or a part
of an organization's knowledge management system.

\subsection{Assessing Knowledge Management
  Success}\label{assessing-knowledge-management-success}

\begin{figure}[!h]
  \centering
  \includegraphics[page=17, trim = 1.5cm 4cm 3cm 4cm, clip, width=\imagewidth]{images/05 - KM.pdf}
\end{figure}

\subsubsection{Project-Oriented Metrics}\label{project-oriented-metrics}

The effective utilization of existing technology is crucial for the
success of a knowledge management initiative. Assessing the success of
such an initiative can be done from various perspectives, starting with
a project-oriented approach. Companies often launch knowledge management
projects, and evaluating their success involves considering factors such
as the resources allocated to the project, the scope of the initiative
(including the number of divisions or units involved), the number of
people engaged, and the longevity of the project. These project-oriented
metrics are particularly significant in the early stages of knowledge
management initiatives, where the focus is on the project itself.

\subsubsection{User Satisfaction and
  Feedback}\label{user-satisfaction-and-feedback}

As time progresses, it is crucial to assess the success of the knowledge
management project. One way to do this is by conducting a survey to
gather feedback from the individuals involved. Positive feedback is
essential because it indicates that people are satisfied with the
project. This satisfaction is important because individuals need to be
willing to contribute their knowledge to the knowledge management
system. If they are not happy, they are unlikely to provide any input or
share their knowledge.

\subsubsection{Technology and Usage
  Assessment}\label{technology-and-usage-assessment}

When assessing the success of knowledge management, companies often want
to evaluate their technologies. They ask questions like: Do we have the
right technologies in place? What technologies do we currently have? To
determine the maturity of their information system from a knowledge
management perspective, companies can conduct a survey and define the
scope of their knowledge management system (KMS). They identify key
functionalities within their ERP that are crucial for knowledge
management and include them as part of their KMS.

In addition to assessing the technologies, it is important to evaluate
the usage of the knowledge management system. Usage is a significant
metric of success because if an application is not being used, it is
clearly not successful. However, it is important to note that just
because an application is being used does not necessarily mean it is
successful from a knowledge management perspective.

\subsubsection{Financial and Efficiency
  Evaluation}\label{financial-and-efficiency-evaluation}

The success of a knowledge management system can be initially driven by
extrinsic incentives, such as monetary rewards or the chance to win
prizes. These incentives encourage people to use the system, resulting
in high usage metrics that may indicate success. However, the true test
of success lies in what happens when these incentives are removed. Will
people continue to use the system?

When launching a new application or service, it is common to provide
incentives to attract users and generate word-of-mouth promotion. But
once the incentives are no longer available, the sustainability of the
system becomes uncertain. It is important to consider the long-term
usage and engagement of users beyond the initial incentive period.

To assess the success of a knowledge management system, it is crucial to
look beyond simple usage metrics, such as the number of clicks. Instead,
focus on the number of active users and contributors. It is essential to
ensure that the number of contributions outweighs the number of
downloads, indicating that people are actively using and benefiting from
the system.

A balanced scorecard of metrics can provide a comprehensive evaluation
of the success of knowledge management. This includes assessing factors
like review cycle time, number of claims, customer satisfaction, and
other financial benefits specific to the organization and its goals. By
evaluating both the financial and efficiency aspects of knowledge
management, a more accurate assessment of its impact can be obtained.

\subsection{Challenges in Knowledge
  Management}\label{challenges-in-knowledge-management}

\begin{figure}[!h]
  \centering
  \includegraphics[page=18, trim = 1.5cm 5.5cm 3cm 4cm, clip, width=\imagewidth]{images/05 - KM.pdf}
\end{figure}

\subsubsection{High Failure Rate and Governance
  Issues}\label{high-failure-rate-and-governance-issues}

When evaluating knowledge management, it is important to consider
multiple indicators rather than relying on just one. One significant
challenge in knowledge management is the high failure rate of projects.
The question then arises: why do these projects fail so frequently? The
main reason is often a governance issue, indicating that the project was
not properly managed.

\subsubsection{Overestimation of IT
  Tools}\label{overestimation-of-it-tools}

One of the challenges in knowledge management is the overestimation of
IT tools. Companies often assume that providing employees with the
necessary tools is enough to ensure their usage. However, this is not
always the case. Simply having the tools does not guarantee their
effective utilization.

Another challenge is the lack of assessment and metrics to measure the
benefits of knowledge management. Without proper evaluation,
organizations may not be able to adjust their work direction to maximize
the benefits of knowledge management.

Additionally, there can be governance issues, such as the absence of
clear accountability and ownership for the knowledge management project.
When multiple departments are involved, it is crucial to have a
dedicated unit or individual responsible for driving the project
forward.

These organizational and technological challenges can result in
knowledge management tools remaining unused, despite their potential to
enhance productivity and efficiency.

\subsubsection{Balancing Extrinsic and Intrinsic
  Motivation}\label{balancing-extrinsic-and-intrinsic-motivation}

\begin{figure}[!h]
  \centering
  \includegraphics[page=19, trim = 1.5cm 3.6cm 1.5cm 3cm, clip, width=\imagewidth]{images/05 - KM.pdf}
\end{figure}

The challenge in knowledge management lies in finding the right balance
between external and internal motivation. External motivation refers to
the use of monetary rewards to encourage people to contribute knowledge
to the system. This is often necessary to kickstart the process and
ensure that the system has valuable knowledge to share. However, relying
solely on external motivation can lead to issues.

\begin{figure}[!h]
  \centering
  \includegraphics[page=20, trim = 1.5cm 4.7cm 1.5cm 4cm, clip, width=\imagewidth]{images/05 - KM.pdf}
\end{figure}

When people are paid to provide knowledge, they may not contribute
because they genuinely believe they have something valuable to share.
Instead, they may simply provide whatever information they have to
receive the monetary incentive. This results in low-quality content in
the knowledge management system. Users who download this content will be
discouraged by its poor quality, leading to a lack of further engagement
with the system.

To address this challenge, it is important to strike a balance between
external and internal motivation. While external incentives can be
effective in the initial phase of knowledge management, it is crucial to
also foster internal motivation. Internal motivation comes from
individuals who genuinely believe they have valuable knowledge to
contribute and are driven by the desire to share it with their
colleagues. By encouraging this intrinsic motivation, the quality of the
content in the knowledge management system can be improved, leading to
greater user engagement.

Finding the right balance between external and internal motivation is
essential for creating a sustainable knowledge creation cycle. It
ensures that contributions to the system are driven by both external
rewards and the genuine desire to share valuable knowledge.

\subsection{Strategies for Effective Knowledge
  Management}\label{strategies-for-effective-knowledge-management}

\subsubsection{Activating the Knowledge Creation
  Cycle}\label{activating-the-knowledge-creation-cycle}

To activate people's willingness to provide knowledge and be recognized
for it, a voting mechanism can be implemented. This mechanism would
assess the quality of the content provided by individuals, similar to
the ``likes'' on Facebook. By receiving positive feedback such as likes,
downloads, or questions from colleagues, individuals would be ranked
higher in the knowledge management system, giving them a sense of
authority and recognition in their field. Conversely, if the content
does not meet a certain threshold of downloads or engagement, it may be
removed from the system. The goal is for individuals to feel
self-realized and authoritative based on the recognition they receive
for their knowledge. By designing a management system that allows
individuals to showcase their expertise and be acknowledged by their
peers, they will be motivated to share high-quality content. This, in
turn, will encourage others to download and engage with the content,
creating a positive feedback loop.

\subsubsection{Adapting Incentive
  Schemas}\label{adapting-incentive-schemas}

This chart highlights the importance of adapting incentive schemas for
effective knowledge management. It emphasizes that external incentives
often fail and provides examples like Siemens ShareNet that can be
researched further. It also emphasizes the need for continuous
adaptation of incentive schemas based on the project phase. This
adaptation requires experience, which the company can gradually acquire
over time.

\subsubsection{Starting from a Need and Managing Projects
  Carefully}\label{starting-from-a-need-and-managing-projects-carefully}

When it comes to implementing knowledge management, it's important to
approach it strategically. Instead of assuming that people will
automatically provide high-quality content, be vigilant and aware that
this may not always be the case. To begin, identify a specific need or
problem that can be addressed through knowledge management. This could
be something that people have expressed a desire for or a topic that
would benefit from shared knowledge. Once the need has been identified,
manage the project carefully, ensuring that all aspects are well-planned
and executed effectively. By starting from a need and managing the
project with care, you can maximize the effectiveness of your knowledge
management efforts.

In conclusion, it is important to minimize the reliance on external
incentives as the primary motivation for knowledge sharing. While small
incentives may be used initially, they should not be the sole driving
force behind people's willingness to share knowledge. It is crucial to
foster a culture where knowledge sharing is valued and encouraged
intrinsically.