\section{Agile - II Lecture}

\subsection{Project Management Methodologies}

\subsubsection{Waterfall vs Agile}

During the morning session, we discussed two project management
methodologies: Waterfall and Agile. Waterfall is a traditional approach
that is still used today, especially for short and well-defined
projects. It is efficient when you have a clear understanding of what
needs to be done. However, for complex projects, it is better to
consider an Agile approach.

Agile emphasizes the importance of having a cohesive team that is
committed to the project's objectives. While each team member may have
their own personal goals, it is crucial for all stakeholders to keep the
main project goal in mind. Agile is not a complicated concept to grasp.
Instead of rigidly fixing the scope, which refers to the list of
functionalities or requirements for the project, Agile allows for
flexibility. The scope can change throughout the project as needed.

In contrast, Waterfall projects typically experience changes in budget
and time when difficulties or obstacles arise during the project
journey. This is because the scope is fixed from the start, and any
deviations require additional resources.

By understanding the differences between Waterfall and Agile, project
managers can choose the most suitable methodology based on the project's
complexity and requirements.

\subsubsection{Team Dynamics and Agile}

When an IT project is running behind schedule, simply adding more people
to the team is not always the best solution. It may seem logical that if
you have work for 10 people, hiring 10 more will speed up the project.
However, this approach rarely leads to a successful recovery. There's a
saying that you can't have a child in one month by having nine mothers.
In other words, adding more people to a late project does not guarantee
faster completion.

Instead, in agile project management, it's easier to please higher-level
management by adjusting the scope of the project. Since they may not be
fully aware of the project's complete scope, you can work with the
client to prioritize the most important features and deliver them by the
promised deadline, even if you're running behind schedule. This means
implementing the higher-priority items from the product backlog and
potentially leaving out less critical functionalities. A skilled product
owner can effectively manage the product backlog, adding and removing
items to help the team achieve their iteration goals.

By focusing on the most important features and adjusting the scope, you
can deliver the first version of the project on time and within the
fixed budget. This approach allows for flexibility while still meeting
the project's objectives.

\subsubsection{Scope Flexibility and Budget}

\begin{figure}[!h]
    \centering
    \includegraphics[page=35, trim = 1.5cm 4.8cm 1.5cm 4.5cm, clip, width=\textwidth]{images/07 - Bruna_agile_1.pdf}
\end{figure}

In traditional Waterfall projects, the scope is typically fixed and
written in contracts. This means that even if there are minor items
missing from the scope, the client may refuse to pay. To address this
issue, it is important to transition from a Waterfall approach to a more
agile methodology. Agile methodologies have gained popularity worldwide,
although only a small percentage of companies (around 18\%) have fully
embraced them. While many people are experimenting with agile
methodologies, only about 20\% have a solid understanding of agile
practices.

\subsubsection{Agile in Practice}

However, despite its benefits, implementing Agile can be challenging,
especially when faced with unique project requirements. There may be
instances where junior team members are unfamiliar with the methodology
or clients prefer a more traditional approach. Each project can present
its own set of obstacles. However, with time and experience, the team
becomes more proficient in Agile practices, leading to increased
efficiency.

\begin{figure}[!h]
    \centering
    \includegraphics[page=36, trim = 1.5cm 4cm 1.5cm 0cm, clip, width=\textwidth]{images/07 - Bruna_agile_1.pdf}
\end{figure}

In Italy, Agile adoption has been slower compared to other countries.
Nevertheless, there are success stories of companies that have embraced
Agile and achieved remarkable results in a short period of time.

\subsubsection{Challenges of Agile Adoption}

The business world has come to realize that adopting agile methodologies
is the way forward. Let's consider a small company with just five
people. After completing your studies, you and your colleagues decide to
start a company and take on a project. You commit to a fixed scope,
fixed price, fixed time approach, and you work tirelessly, even
sacrificing sleep to meet deadlines. However, as the company grows and
you hire new employees who are not founders, they may not be willing to
work under such intense pressure.

This is where agile comes in. By adopting an agile approach, you can
reduce the risks for your company. Instead of promising a fixed scope,
you focus on leveraging your team's capabilities and talents. You work
closely with a product owner, who could be your client, and maintain a
transparent relationship with them. This allows for better productivity
and higher quality work in a more relaxed environment. By continuously
listening to customer feedback, you can ensure higher customer
satisfaction. Ultimately, this creates a win-win relationship between
your customer and your company, resulting in a higher return on
investment.

Agile has become a widely used methodology over the years, and its
success is supported by various statistics. It offers a more flexible
and efficient way of working, which is why it has gained popularity in
the business world.

\subsubsection{Agile and Contracting}

Implementing agile methodologies can be challenging, as it requires a
shift in mindset rather than just technical knowledge. One of the most
well-known agile methodologies is explained later in this section. Agile
concepts are not technical in nature; they are about adopting a specific
approach to work. While it may be easier for some to start using agile
methodologies, it is important to be precise and follow the rules while
respecting others involved in the project. There are certain
restrictions and agreements that must be adhered to during the project.
However, there may be times when fatigue sets in and these principles
are not followed.

For individuals who have spent their entire careers working with
different types of relationships, such as service providers and
companies, or startups working for clients, these new agile dynamics may
be unfamiliar. In the past, companies would often publish rankings,
known as ``Gara'' in Italian, to select the company that would implement
a project. Participating in this ranking involved interviews and
declaring the daily cost of developers, among other evaluations. The
ranking was based on factors such as the promised delivery date and
budget. This system encouraged overestimating effort, underpaying team
members, and working within fixed scopes, prices, and budgets. The
company ranked highest in the list would be chosen. In the public
sector, this ranking system is mandatory.

\subsubsection{Organizational Impact of Agile}

Choosing a team based solely on cost rather than skill is a common but
flawed approach that often leads to unsuccessful outcomes. Implementing
Agile in a large organization is challenging due to the organizational
repercussions it brings. With Agile, decision-making power shifts from
higher-level managers to the teams actually doing the work, bypassing
middle managers. This change can be met with resistance from those who
are no longer needed or who must take on new roles and responsibilities.
Some individuals may be hesitant to embrace Agile because it requires
them to step out of their comfort zones and develop new skills.
Additionally, employees nearing retirement may be reluctant to start a
new career path. These factors contribute to the difficulty of
introducing Agile as an organizational revolution in companies.

\begin{figure}[!h]
    \centering
    \includegraphics[page=42, trim = 0cm 5.4cm 2cm 6.2cm, clip, width=\textwidth]{images/07 - Bruna_agile_1.pdf}
\end{figure}

Many large companies have started implementing Agile methodologies
gradually, beginning with side projects involving younger employees.
Over time, as more projects are conducted using these methodologies,
higher-level managers begin to realize the benefits. However, it's
important to note that the initial projects may not always go smoothly
due to factors such as resistance to change or lack of experience.
Despite this, there are numerous books available that discuss the
organizational transformations happening within companies.

\subsection{Scrum Framework}

\subsubsection{Scrum Overview}

\begin{figure}[!h]
    \centering
    \includegraphics[page=4, trim = 2cm 6cm 2.5cm 5.9cm, clip, width=\textwidth]{images/08 - Bruna_scrum.pdf}
\end{figure}

Now, let's focus on the Scrum methodology itself. The Scrum framework
can be summarized in a single slide, which I will present shortly.
Before we dive into the details, it's important to understand the core
principles outlined in the Agile Manifesto. These principles emphasize
the importance of individuals and interactions over processes and tools.
Instead of treating developers as replaceable units of production, the
focus is on collaboration and communication between team members.
Additionally, working software is prioritized over comprehensive
documentation, as the latter tends to become outdated quickly. Customer
collaboration is crucial, as it helps build trust and avoids unnecessary
conflicts. Finally, the ability to respond to change is valued more than
strictly following a plan, as projects rarely go exactly as planned.

While Gantt diagrams can be useful for communicating project intentions,
it's important to acknowledge their limitations. They provide a visual
representation of the project plan, but it's crucial to recognize that
the plan may need to be adjusted as the project progresses. Estimations
can be uncertain, and unexpected changes can occur, leading to
variations in scope, time, and budget. Therefore, it's essential to
maintain flexibility and adaptability throughout the project.

\begin{figure}[!h]
    \centering
    \includegraphics[page=5, trim = 1.5cm 2.7cm 1.8cm 4.5cm, clip, width=\textwidth]{images/08 - Bruna_scrum.pdf}
\end{figure}

Now, let's explore some key aspects of the Scrum framework. One
important principle is that the best architectures, requirements, and
designs emerge from self-organizing teams. This means that the team
members themselves are best equipped to determine how to implement a
solution. It's not ideal for a product owner or customer to impose
technical solutions without valid business or functional reasons.

In Scrum, working software is considered the primary measure of
progress. Instead of relying solely on time sheets or other
organizational tools, regular demos are conducted to showcase the
implemented functionality. This allows for continuous feedback and
ensures that progress is visible to all stakeholders.

Another crucial aspect is the daily collaboration between business
people and developers throughout the project. This eliminates the
traditional approach of business people providing requirements and then
disengaging until the project is complete. Instead, regular interactions
and feedback sessions are encouraged to maintain a strong partnership.

It's important to embrace changes in the product backlog, which is the
list of tasks to be completed. Clients may prioritize certain features
or request new ones, and this dynamic backlog ensures that the project
remains alive and responsive to evolving needs.

\begin{figure}[!h]
    \centering
    \includegraphics[page=8, trim = 0cm 4.4cm 0cm 5.2cm, clip, width=\textwidth]{images/08 - Bruna_scrum.pdf}
\end{figure}

The term ``Scrum'' originates from rugby, where it refers to the
collective effort of the team pushing towards a common goal. In the
Scrum methodology, there are three roles, five events, and three
artifacts that form the framework. These elements provide structure and
guidance for effective project management.

By understanding and implementing the Scrum framework, companies can
benefit from improved collaboration, adaptability, and a focus on
delivering working software.

The Scrum framework consists of three key roles: the Product Owner, the
Scrum Master, and the Developers. The Product Owner is responsible for
managing the project and has similar responsibilities to a project
manager or leader. We will delve into their specific responsibilities
later. The Scrum Master, contrary to the name, is not a master over
others but rather a servant leader. Their main purpose is to support and
assist the team in their work. The Developers, who can include various
disciplines such as designers, business analysts, and other skilled
individuals, are responsible for developing the product.

\begin{figure}[!h]
    \centering
    \includegraphics[page=10, trim = 1.2cm 3.7cm 1.5cm 5.5cm, clip, width=\textwidth]{images/08 - Bruna_scrum.pdf}
\end{figure}

In addition to the roles, there are five important events in the Scrum
framework. These events include Backlog Refinement, Sprint Planning,
Daily Scrum, Sprint Review, and Retrospective. Each event serves a
specific purpose and contributes to the overall success of the project.

\begin{figure}[!h]
    \centering
    \includegraphics[page=11, trim = 1.2cm 3.7cm 1.5cm 5.5cm, clip, width=\textwidth]{images/08 - Bruna_scrum.pdf}
\end{figure}

Furthermore, there are three artifacts in Scrum. The first is the
Product Backlog, which is a list of tasks that need to be completed. The
project is considered finished when the Product Backlog is empty. The
second artifact is the Sprint Backlog, which is a subset of the Product
Backlog and contains the tasks to be completed during a specific sprint.
Finally, the third artifact is the working software itself, which is the
tangible outcome of the team's efforts.

Now, let's explore the details of each Scrum role.

\subsubsection{Scrum Roles}

\begin{figure}[!h]
    \centering
    \includegraphics[page=15, trim = 0.5cm 4.8cm 1.5cm 2.7cm, clip, width=\textwidth]{images/08 - Bruna_scrum.pdf}
\end{figure}

The product owner plays a crucial role in maximizing the value of the
product. They define the product vision and gather requirements, which
may be delegated to others. The product owner is responsible for
managing the product backlog, ensuring it is transparent and clear.
While they don't need to know every detail, they are accountable for
making sure the backlog is understandable to all team members. The
product owner also collaborates with the team and answers
product-related questions during Scrum ceremonies.

\begin{figure}[!h]
    \centering
    \includegraphics[page=17, trim = 0.5cm 5.5cm 1.5cm 2.7cm, clip, width=\textwidth]{images/08 - Bruna_scrum.pdf}
\end{figure}

The development team consists of cross-functional members, typically
ranging from three to nine people. In larger projects, multiple Scrum
teams may be created, with each team having its own product owner. The
product owners collaborate with each other and receive guidance from a
higher-level product owner. However, some argue that it is better to
have a single product owner, even with multiple teams, as it can be
challenging for one person to have knowledge of all aspects of the
project.

Within the development team, there are no recognized roles or titles.
Even if someone has more experience, they are considered equal to other
team members. However, due to their experience, they may be sought out
for advice or guidance. The self-organizing nature of the team allows
for individual growth and decision-making.

\begin{figure}[!h]
    \centering
    \includegraphics[page=18, trim = 0.5cm 5cm 1.5cm 2.7cm, clip, width=\textwidth]{images/08 - Bruna_scrum.pdf}
\end{figure}

The Scrum Master is responsible for facilitating the Scrum process and
ensuring its effectiveness. They act as a coach and support the team in
removing any obstacles they may face. The Scrum Master may have
non-technical skills and can analyze team performance and track project
progress using metrics like burn-down diagrams. Their role is separate
from the product owner to maintain a clear distinction between project
scope and team needs.

The Scrum Master's role is important because they have a broader view of
the project and can provide guidance without being directly involved in
the day-to-day development. This separation allows them to protect the
team from any changes in functionality or priorities that may conflict
with the agreed-upon scope. The Scrum Master acts as a coach, similar to
how a coach in a sports match does not play alongside the players.

In Agile methodologies, the product owner, development team, and Scrum
Master each have their own responsibilities and work together to ensure
the success of the project. The team commits to a piece of the backlog
during a sprint, which typically lasts one to four weeks, and the Scrum
Master ensures that all rules and guidelines are followed by all team
members and the product owner.

\subsubsection{Scrum Events}

\begin{figure}[!h]
    \centering
    \includegraphics[page=20, trim = 2cm 5cm 1cm 5.5cm, clip, width=\textwidth]{images/08 - Bruna_scrum.pdf}
\end{figure}

Every Sprint in the Scrum framework has a Sprint goal that is
established during the Sprint planning. The typical duration of a Sprint
is two weeks, and it is generally recommended not to change the duration
once it is set. However, there may be cases where a Sprint needs to be
aborted. For example, if it is discovered in the middle of the Sprint
that the development of a particular feature is not useful at all, it is
better to stop the development rather than continue. It is important to
avoid making changes to the Sprint backlog during the Sprint to minimize
interruptions.

During the Sprint planning, items are selected from the product backlog
and added to the Sprint backlog, which is the list of tasks to be
completed during the Sprint. The criteria for selecting items from the
product backlog is simple: items are taken from the top of the backlog
and placed in the Sprint backlog. If there are any technical
requirements that are necessary for the implementation of functional
items, the team can communicate this to the product owner. The product
owner usually agrees to include these technical requirements in the
backlog because they are important for the functionality. The Sprint
backlog is considered complete once all the necessary items have been
selected.

\begin{figure}[!h]
    \centering
    \includegraphics[page=22, trim = 0.5cm 6cm 1.5cm 2.7cm, clip, width=\textwidth]{images/08 - Bruna_scrum.pdf}
\end{figure}

The Sprint planning meeting typically lasts from four to eight hours.
During this meeting, the product owner explains the functionalities to
the team, and the team estimates the effort required to implement them.
This meeting also provides an opportunity for the team to discuss and
contribute their ideas and expertise to the implementation of the
functionalities. This collaboration helps distribute knowledge among
team members and ensures that the work can continue even if someone is
absent or leaves the team.

One of the benefits of the Sprint planning meeting is that it allows the
team to estimate the work and determine how much can be included in the
Sprint backlog. If the team realizes that they cannot complete all the
planned items within the Sprint, they can remove some items from the
backlog and prioritize the most important ones. Once the team commits to
a scope for the Sprint, they are expected to complete it without any
interruptions or delays.

\subsubsection{Sprint Planning}

During the Sprint, if any impediments or issues arise, they can be
discussed and addressed in the daily Scrum meeting. The product owner
can be informed about any impediments that may affect the completion of
certain items, and adjustments can be made if necessary. However, it is
important for the team to stay committed to the Sprint goal and complete
the agreed-upon scope.

\begin{figure}[!h]
    \centering
    \includegraphics[page=23, trim = 0.5cm 4.8cm 1.5cm 2.7cm, clip, width=\textwidth]{images/08 - Bruna_scrum.pdf}
\end{figure}

The Sprint planning meeting is the first part of the Sprint, where the
product backlog is used as input to create the Sprint backlog. The
team's capacity and velocity are taken into account to determine the
amount of work that can be included in the Sprint. The output of the
Sprint planning meeting is the Sprint backlog, which includes the
selected items and their estimated effort.

\begin{figure}[!h]
    \centering
    \includegraphics[page=24, trim = 0.5cm 5.5cm 1.5cm 2.7cm, clip, width=\textwidth]{images/08 - Bruna_scrum.pdf}
\end{figure}

In the past, it was common to break down each item in the Sprint backlog
into subtasks during the Sprint planning meeting. However, this practice
is now less common and is only done if necessary for better organization
or estimation. It is more common for developers to create internal tasks
within each item if they need more granularity in their work. This
allows for better estimation and helps confirm whether the team can
commit to the Sprint goal.

The focus during the Sprint planning meeting is on the most important
items in the product backlog. Time is not wasted on estimating items
that are not currently being considered for implementation. This
approach is lean and agile, as effort is only spent when necessary and
on the most important tasks. Detailed design work is postponed until it
is needed, and the team can start preparing for the next iteration by
working on the detailed analysis of upcoming tasks.

\begin{figure}[!h]
    \centering
    \includegraphics[page=25, trim = 0.5cm 4.4cm 2cm 2.7cm, clip, width=\textwidth]{images/08 - Bruna_scrum.pdf}
\end{figure}

In summary, the Sprint planning meeting is an important part of the
Scrum framework, where the Sprint backlog is created based on the
product backlog. The team estimates the effort required for each item
and commits to completing a specific scope. The daily Scrum meeting is
another crucial meeting in Scrum, where the team discusses progress,
impediments, and adjusts their plan if necessary.

\subsubsection{Daily Scrum}

\begin{figure}[!h]
    \centering
    \includegraphics[page=26, trim = 0.5cm 5.5cm 1.5cm 2.7cm, clip, width=\textwidth]{images/08 - Bruna_scrum.pdf}
\end{figure}

In the past, when there was less remote working and more in-person
collaboration, we used a specific pattern for our meetings. We would
gather in a circle with all nine team members in the same room. To
regulate the discussion, we introduced a simple trick: we passed around
a small ball, and only the person holding the ball had the right to
speak. This helped us manage the tendency for everyone to talk over each
other, which is common in Italian culture. If someone wanted to speak,
they would raise their hand and request the ball. Then, they would share
three things: what they had done yesterday, what they planned to do
today, and if they had any impediments. Each person had a maximum of two
minutes to share their updates. This format ensured that the meeting
lasted no longer than 15 minutes.

This meeting format had a powerful impact on our projects. It allowed us
to quickly identify if someone was facing difficulties or impediments in
their work. No one could hide their challenges, as they would be brought
up during the meeting, giving the team an opportunity to offer help or
for the Scrum Master to address any organizational issues. This open
communication prevented situations where team members struggled for days
without anyone knowing. It was a very effective way to cut waste from
our projects.

Additionally, the daily meeting provided a valuable learning opportunity
for junior team members who were unfamiliar with the technologies being
used. They could ask questions and receive guidance from more
experienced team members. This helped them learn and contribute to the
project more effectively.

However, one challenge we faced with the daily meeting was that
sometimes developers would start to engage in lengthy discussions,
exceeding the allocated two minutes per person. This resulted in
meetings that lasted longer than the intended 15 minutes. While
occasional deviations were acceptable, it was important to keep the
meetings concise. The Scrum Master was responsible for ensuring the
meeting stayed on track and was completed within the agreed-upon time.

\begin{figure}[!h]
    \centering
    \includegraphics[page=27, trim = 0.5cm 4.2cm 2cm 2.7cm, clip, width=\textwidth]{images/08 - Bruna_scrum.pdf}
\end{figure}

In the past, when in-person attendance was expected, we implemented a
system to encourage punctuality. If someone arrived late to the daily
meeting, they would contribute a small amount of money (such as one or
two euros) to a collective fund. At the end of the project, we would use
the collected money to organize a pizza party for the team. This
practice helped incentivize everyone to be more punctual and adhere to
the schedule. However, with the shift to remote working, we have
adjusted our approach. Now, if someone is more than two minutes late, we
start the meeting without them, ensuring that we have enough time to
discuss the important topics: the sprint review and the retrospective.

\subsubsection{Sprint Review and Retrospective}

\begin{figure}[!h]
    \centering
    \includegraphics[page=28, trim = 0.5cm 5cm 2cm 2.7cm, clip, width=\textwidth]{images/08 - Bruna_scrum.pdf}
\end{figure}

\begin{figure}[!h]
    \centering
    \includegraphics[page=30, trim = 0.5cm 6cm 1.5cm 2.7cm, clip, width=\textwidth]{images/08 - Bruna_scrum.pdf}
\end{figure}

In conclusion, the sprint planning takes place at the beginning of the
iteration, followed by daily work throughout the week. At the end of the
iteration, there is a sprint review where the team demonstrates their
progress to key stakeholders. During this review, the product owner has
the responsibility to accept or reject the deliverables. This
transparent process helps avoid conflicts between developers and project
managers. After the review, the team holds a sprint retrospective to
reflect on their mistakes and successes during the iteration. They work
together to find ways to improve their work and avoid repeating the same
mistakes in the future. This continuous improvement philosophy is a
crucial part of the agile methodology. If you have any questions, feel
free to ask next Monday. Thank you for your attention.
